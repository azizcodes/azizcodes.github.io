% Document écrit par 'SIFI Khedidja'(sifikhedidja@gmail.com)
% Compilé avec XELATEX

\documentclass[12pt]{article}
\usepackage[left=2cm,right=2cm,top=1.8cm,bottom=1.8cm]{geometry}
\usepackage{fancyhdr}
\usepackage{graphicx}
\usepackage[dvipsnames]{xcolor}
\usepackage{pgfornament}
\usepackage{polyglossia}
\usepackage[]{quoting}

\setdefaultlanguage[calendar=gregorian,locale=kuwait]{arabic}
\setotherlanguage{french}

\newfontfamily\arabicfont[Script=Arabic,Scale=1.4]{Amiri}

\definecolor{fondpaille}{cmyk}{0,0,0.1,0}

\pagestyle{fancy}
\title{تقرير عن اللغة العربية \\
\large  من إعداد المهندس عبدالعزيز}

\renewcommand{\headrulewidth}{1pt}
\renewcommand{\footrulewidth}{1pt}
\rhead{\textbf{\LaTeX}}

\chead{
              \color{violet}{\pgfornament[scale=.4]{16}}
              \color{black}{من إعداد: عبدالعزيز }
%              \color{violet}{\pgfornament[scale=.4]{15}}
}

      
\lhead{\textbf{عبدالعزيز}}

\begin{document}
\maketitle
\tableofcontents


\section*{مقدمة}
أتوجه بالشكر و التقدير لكل من دعم مسيرتي و أدعو الله أن ينتقم من كل من خذلني.


\section{قليل من الشعر لعيونك}

يقول الشاعر الكبير:


لولا الهوى لم ترق دمعًا على طلل... و لا أرقت لذكر البان و العلم


صدق. نعم صدق. و يقول طلال:


باكتب لك رسالة حب... تعبر عن عذاب القلب

باكتب لك رسالة حب... تعبر عن عذاب القلب


سأتكلم عن طريقة الاستخدام لاحقًا.


ما فيه أحد مرتاح... ما فيه أحد مرتاح

\section{عن اللغة العربية - رحمها الله تعالى}

تم نسخ هذا المحتوى من ويكيبيديا لتقييم جودة الطباعة.

اللُّغَة العَرَبِيّة هي أكثر اللغات السامية تحدثاً وإحدى أكثر اللغات انتشاراً في العالم، يتحدثها أكثر من 467 مليون نسمة،(1) ويتوزع متحدثوها في الوطن العربي، بالإضافة إلى العديد من المناطق الأخرى المجاورة كالأهواز وتركيا وتشاد ومالي والسنغال وإرتيريا وإثيوبيا وجنوب السودان وإيران. وبذلك فهي تحتل المركز الرابع أو الخامس من حيث اللغات الأكثر انتشاراً في العالم، واللغة الرابعة من حيث عدد المستخدمين على الإنترنت. اللغةُ العربيةُ ذات أهمية قصوى لدى المسلمين، فهي عندَهم لغةٌ مقدسة إذ أنها لغة القرآن، وهي لغةُ الصلاة وأساسيةٌ في القيام بالعديد من العبادات والشعائرِ الإسلامية.

العربيةُ هي أيضاً لغة شعائرية رئيسية لدى عدد من الكنائس المسيحية في الوطن العربي، كما كُتبَت بها كثير من أهمِّ الأعمال الدينية والفكرية اليهودية في العصور الوسطى. ارتفعتْ مكانةُ اللغةِ العربية إثْرَ انتشارِ الإسلام بين الدول إذ أصبحت لغة السياسة والعلم والأدب لقرون طويلة في الأراضي التي حكمها المسلمون. وللغة العربية تأثير مباشر وغير مباشر على كثير من اللغات الأخرى في العالم الإسلامي، كالتركية والفارسية والأمازيغية والكردية والأردية والماليزية والإندونيسية والألبانية وبعض اللغات الإفريقية الأخرى مثل الهاوسا والسواحيلية والتجرية والأمهرية والصومالية، وبعض اللغات الأوروبية وخاصةً المتوسطية كالإسبانية والبرتغالية والمالطية والصقلية؛ ودخلت الكثير من مصطلحاتها في اللغة الإنجليزية واللغات الأخرى، مثل أدميرال والتعريفة والكحول والجبر وأسماء النجوم. كما أنها تُدرَّس بشكل رسمي أو غير رسمي في الدول الإسلامية والدول الإفريقية المحاذية للوطن العربي.

العربية لغةٌ رسمية في كل دول الوطن العربي إضافة إلى كونها لغة رسمية في تشاد وإريتريا. وهي إحدى اللغات الرسمية الست في منظمة الأمم المتحدة، ويُحتفل باليوم العالمي للغة العربية في 18 ديسمبر كذكرى اعتماد العربية بين لغات العمل في الأمم المتحدة. وفي سنة 2011 صنفت بلومبيرغ بيزنس ويك اللغة العربية في المرتبة الرابعة من حيث اللغات الأكثر فائدة في الأعمال التجارية على مستوى العالم. "وفي 2013 نشر المجلس الثقافي البريطاني تقريرًا مفصلاً عن اللغات الأكثر طلباً في المملكة المتحدة تحت عنوان "لغات المستقبل" وتبين ان العربية تحتل المرتبة الثانية على مستوى العالم وفي 2017 احتلت المرتبة الرابعة. فيما يخص اللغات الأكثر جنيا للأرباح في بريطانيا تاتي العربية في المرتبة الثانية وفقا للمنظمة.واللغة العربية من أغزر اللغات من حيث المادةِ اللغوية، فعلى سبيل المثال يحوي معجم لسان العرب لابن منظور من القرن الثالث عشر أكثر من 80 ألف مادة، بينما في اللغة الإنجليزية فإن قاموس صموئيل جونسون - وهو من أوائل من وضع قاموساً إنجليزياً من القرن الثامن عشر- يحتوي على 42 ألف كلمة.

تحتوي اللغة العربية 28 حرفاً مكتوباً. ويرى بعضُ اللغويين أنه يجب إضافة حرف الهمزة إلى حروف العربية، ليصبحَ عدد الحروف 29. تُكتب العربية من اليمين إلى اليسار - ومثلها اللغة الفارسية والعبرية على عكس كثير من اللغات العالمية - ومن أعلى الصفحة إلى أسفلها.

تنتمي اللغة العربية إلى أسرة اللغات السامية المتفرعة من مجموعة اللغات الإفريقية الآسيوية. وتضم مجموعة اللغات السامية لغات حضارة الهلال الخصيب القديمة، مثل الأكادية والكنعانية والآرامية واللغة الصيهدية (جنوب الجزيرة العربية) واللغات العربية الشمالية القديمة وبعض لغات القرن الإفريقي كالأمهرية. وعلى وجه التحديد، يضع اللغويون اللغة العربية في المجموعة السامية الوسطى من اللغات السامية الغربية.

والعربية من أحدث هذه اللغات نشأة وتاريخاً، ولكن يعتقد البعض أنها الأقرب إلى اللغة السامية الأم التي انبثقت منها اللغات السامية الأخرى، وذلك لاحتباس العرب في جزيرة العرب فلم تتعرض لما تعرضت له باقي اللغات السامية من اختلاط.
ولكن هناك من يخالف هذا الرأي بين علماء اللسانيات، حيث أن تغير اللغة هو عملية مستمرة عبر الزمن والانعزال الجغرافي قد يزيد من حدة هذا التغير حيث يبدأ نشوء أيّة لغة جديدة بنشوء لهجة جديدة في منطقة منعزلة جغرافياً. بالإضافة لافتراض وجود لغة سامية أم لا يعني وجودها بالمعنى المفهوم للغة الواحدة بل هي تعبير مجازي قصد به الإفصاح عن تقارب مجموعة من اللغات فقد كان علماء اللسانيات يعتمدون على قرب لغة وعقلية من يرونه مرشحاً لعضوية عائلة اللغات السامية وبُنيت دراساتهم على أسس جغرافية وسياسية وليس على أُسس عرقية ولا علاقة لها بنظرة التوراة لأبناء سام وكثرة قواعد اللغة العربية ترجح أنها طرأت عليها في فترات لاحقة وأنها مرت بأطوار عديدة مما يضعف فرضية أن هذه اللغة أقرب لما عُرف اصطلاحاً باللغة السامية الأم هذه، ولا توجد لغة في العالم تستطيع الادعاء أنها نقية وصافية من عوامل ومؤثرات خارجية.

هنالك العديد من الآراء حول أصل العربية لدى قدامى اللغويين العرب، منها أن اللغة العربية أقدم من العرب أنفسهم فقالوا أنها لغة آدم في الجنة، ولعب التنافس القبلي في عصر الخلافة العباسية دوراً كبيراً في نُشوء هذه النظريات، فزعم بعضهم أن يعرب بن قحطان كان أول من تكلم هذه العربية، وفريق ذهب أن إسماعيل هو أول من تكلم بها، واستشهدوا لقولهم هذا بالقرآن الكريم والأحاديث النبوية والواقع الحي ،(2)، الدراسات العلمية الحديثة أكدت أن العربية والعبرية والسيرانية (اللغات السامية الوسطى) تنحدر من أصل لغوي واحد، وعثر في مواضع مُتعدّدة في شمال شبه الجزيرة العربية كذلك على كتابات قديمة بلغات متباينة اختلاف بسيط عن عربية القرآن أو الشعر الجاهلي ولم يهتم اللغويون العرب القدماء باللغات السامية الجنوبية بشقيها الشرقي والغربي واعتبروها لغات "رديئة"، فقد اعتبروا اللغة العربية لغة قريش هي الأصل رغم أن تلك اللغات السامية الجنوبية قد تكون أقدم من العربية التي تكلمت بها قريش. وبعضهم كان يرى أن دراسة وبحث تلك اللغات واللهجات مضيعة للوقت وإحياءً للجاهلية فقد كانوا مُدركين أن ألسنة العرب متباينة ومختلفة، فقد قال محمد بن جرير الطبري:
« كانت العرب وإن جمع جميعها اسم أنهم عرب، فهم مختلفو الألسن بالبيان متباينو المنطق والكلام»


\section{لغة قريش}

ومنهم من يرى أنها لغة قريش خاصة ويؤيد هذا الرأي أن أقدم النصوص المتوفرة بهذه اللغة هو القرآن والنبي محمد قُرشي وأول دعوته كانت بينهم وهو الرأي الذي أجمع عليه غالب اللغويين العرب القدماء
ومنهم من يرى أنها لهجة عربية تطورت في مملكة كندة في منتصف القرن السادس الميلادي بسبب إغداق ملوك تلك المملكة المال على الشعراء فأدى لتنافسهم وتوحد لهجة شعرية بينهم وهم أقدم من قريش وأيد ذلك العديد من المستشرقين فرجّحوا وجود ما أسموه بـ"اللغة العالية" وهي لغة شعرية خاصة بالإضافة للهجات محلية فاعتبروا تلك اللغة لغة رفيعة تظهر مدارك الشاعر وثقافته أمام الملك

والرأي القائل أنها لغة قريش أقوى لأن أقدم النصوص بهذه اللغة هو القرآن فالشعر الجاهلي، إن كان جاهليًا حقًا، دُوّن بعد الإسلام ولا يملك الباحثون نسخة أصلية لمُعلّقة أو قصيدة جاهلية ليُحدّد تاريخها بشكل دقيق.


\section{قريش قريش قريش}

توجه العلماء الأقدمون إلى القول بأن مكة كانت "مهوى أفئدة العرب" وأنهم كانوا يعرضون لغتهم على قريش وأن تلك القبيلة كانت تختار الأصلح فتأخذه وتترك الرديء حتى غلبت لغتهم شبه الجزيرة بكاملها قبل الإسلام.
يٌفنّد هذا الرأي الكتابات التي لا تبعد عن الإسلام بكثير وهي مكتوبة بلهجة مختلفة عن عربية القرآن فلم يُعثر على دليل أو أثر أن أحدًا من العرب قٌبيل الإسلام دوّن بهذه اللغة وأقرب الكتابات لها هي خمسة نصوص كُتبت بعربية نبطية وهي لغة مُتحكمة في أسلوبها وقواعدها والكثرة الغالبة من كلماتها تمنعها أن تعد في عداد عربية القرآن. وسيادة اللغة ترتبط غالبًا بسيادة سياسية ولا يوجد دليل قطعي على هذه السيادة القُرشية على القبائل قبل الإسلام فقد كانت العرب قبل الإسلام تعدّ قريشًا تُجّارًا وليسوا مقاتلين ويُرجّح عدد من الباحثين أن كل الوارد أنها لهجة قريش كان من باب تفضيل النبي محمد أو هو نتاج التنافس بين الأنصار والمهاجرين، ولم يرد في القرآن أنها لغة قريش بل وردت آيات تحدي أن يأتوا بمثله فهذا التحدي أن يأتوا بمثله وبنفس لسانه "العربي المبين " دليل أنه أكمل الألسنة العربية وليس لسان بعض العرب على غيرهم بل إن المسلمين يعدّون القرآن معجزة بحد ذاتها. أما أصل هذه اللغة ففيه اختلاف بين العلماء فكل الوارد عن أنها لهجة قريش سببه عدم العثور على أثر يسبق الإسلام مُدوّن بهذه اللغة ومصدر الباحثين الوحيد هو المصادر الإسلامية لاستنباط رأي علمي مقبول.


\section{ايماكس}

تم تحرير هذا النص بواسطة برنامج ايماكس. و تمت طباعته باستخدام xetex


\section{نجرب الاقتباس}

قبل الاقتباس:
\begin{quoting}
توجه العلماء الأقدمون إلى القول بأن مكة كانت "مهوى أفئدة العرب" وأنهم كانوا يعرضون لغتهم على قريش وأن تلك القبيلة كانت تختار الأصلح فتأخذه وتترك الرديء حتى غلبت لغتهم شبه الجزيرة بكاملها قبل الإسلام.
يٌفنّد هذا الرأي الكتابات التي لا تبعد عن الإسلام بكثير وهي مكتوبة بلهجة مختلفة عن عربية القرآن فلم يُعثر على دليل أو أثر أن أحدًا من العرب قٌبيل الإسلام دوّن بهذه اللغة وأقرب الكتابات لها هي خمسة نصوص كُتبت بعربية نبطية وهي لغة مُتحكمة في أسلوبها وقواعدها والكثرة الغالبة من كلماتها تمنعها أن تعد في عداد عربية القرآن. وسيادة اللغة ترتبط غالبًا بسيادة سياسية ولا يوجد دليل قطعي على هذه السيادة القُرشية على القبائل قبل الإسلام فقد كانت العرب قبل الإسلام تعدّ قريشًا تُجّارًا وليسوا مقاتلين ويُرجّح عدد من الباحثين أن كل الوارد أنها لهجة قريش كان من باب تفضيل النبي محمد أو هو نتاج التنافس بين الأنصار والمهاجرين، ولم يرد في القرآن أنها لغة قريش بل وردت آيات تحدي أن يأتوا بمثله فهذا التحدي أن يأتوا بمثله وبنفس لسانه "العربي المبين " دليل أنه أكمل الألسنة العربية وليس لسان بعض العرب على غيرهم بل إن المسلمين يعدّون القرآن معجزة بحد ذاتها. أما أصل هذه اللغة ففيه اختلاف بين العلماء فكل الوارد عن أنها لهجة قريش سببه عدم العثور على أثر يسبق الإسلام مُدوّن بهذه اللغة ومصدر الباحثين الوحيد هو المصادر الإسلامية لاستنباط رأي علمي مقبول.
\end{quoting}
بعد الاقتباس.

و أدرك شهرزاد الصباح فسكتت عن الكلام المباح.

\end{document} 
